\documentclass[a4paper,10pt]{article}

%A Few Useful Packages
\usepackage{marvosym}
\usepackage{fontspec} 					%for loading fonts
\usepackage{xunicode,xltxtra,url,parskip} 	%other packages for formatting
\RequirePackage{color,graphicx}
\usepackage[usenames,dvipsnames]{xcolor}
%\usepackage[big]{layaureo} 				%better formatting of the A4 page
\usepackage{fullpage}
% an alternative to Layaureo can be ** \usepackage{fullpage} **
\usepackage{supertabular} 				%for Grades
\usepackage{titlesec}					%custom \section

%Setup hyperref package, and colours for links
\usepackage{hyperref}
\definecolor{linkcolour}{rgb}{0,0.2,0.6}
\hypersetup{colorlinks,breaklinks,urlcolor=linkcolour, linkcolor=linkcolour}

%FONTS
\defaultfontfeatures{Mapping=tex-text}
%\setmainfont[SmallCapsFont = Fontin SmallCaps]{Fontin}
%%% modified for Karol Kozioł for ShareLaTeX use
\setmainfont[
SmallCapsFont = Fontin-SmallCaps.otf,
BoldFont = Fontin-Bold.otf,
ItalicFont = Fontin-Italic.otf
]
{Fontin.otf}

\titleformat{\section}{\Large\scshape\raggedright}{}{0em}{}[\titlerule]
\titlespacing{\section}{0pt}{3pt}{3pt}

\usepackage[a4paper,total={7in,10.5in}]{geometry}


\hyphenation{im-pre-se}

\pagenumbering{arabic}
\usepackage{longtable}

\begin{document}

\pagestyle{plain} % non-numbered pages


%\font\fb=''[cmr10]'' %for use with \LaTeX command

%--------------------TITLE-------------
\par{\centering
		{\Huge Georgios Chr. \textsc{Christodoulis}
	}\bigskip\par}

%--------------------SECTIONS-----------------------------------
%Section: Personal Data

\section{Personal Information}

\begin{tabular}{rl}
    \textsc{Place and Date of Birth:} & Patras, Greece  | $25$ August $1990$ \\
    \textsc{Address:}   & I.Sechou $2$, $11524$, Athens, Greece \\
    \textsc{Phone:}     & $+30$ $697$ $85$ $03$ $372$\\
    \textsc{email:}     & \href{mailto:gchristodoulis@gmail.com}{gchristodoulis@gmail.com}
\end{tabular}

%Section: Education
\section{Education}
\begin{tabular}[l]{r|p{1400pt}}	
\textsc{Current} & \textbf{Diploma }($5$-year degree)\\
\textit{September $2008$} & School of Electrical and Computer Engineering,\\
\textit{to}\textsc{(Estim)} &\normalsize\emph {National Technical University of Athens}, Greece\\
\textit{June $2015$} &Thesis: "Performance modeling and Prediction for the communication of parallel applications"\\
&\emph{Thesis Committee}: Prof Nektarios Koziris, Prof. Georgios Goumas\\
&\emph{Majors}: Computer Systems, Computer Software\\
&\emph{Minors}: Telecommunication Systems and Computer Networks, Electronics-Circuits-Materials\\\multicolumn{2}{c}{} \\
\textit{September $2005$} & \textbf{General Lyceum }(Upper Secondary School)\\
\textit{June $2008$} & 2nd General Experimental Lyceum of Athens, Greece\\\multicolumn{2}{c}{}
\end{tabular}
\section{Thesis Abstract}
One of the main challenges in the field of supercomputing is the
development of fast and accurate performance models.
Performance prediction is a powerful tool for the efficient deployment of
modern high-end systems, performance optimization of applications and
performance portability. Parallel applications, running
on high core counts, commonly fail to scale up accordingly,
suffering from communication overheads. In this thesis,
we search for \emph{application} and \emph{topology-related}
metrics of \emph{communication performance} and study their relation
to \emph{network congestion effects}, through benchmarking. Combining
this knowledge, we will attempt to \emph{predict performance} and \emph{provide
generic optimization guidelines} at runtime for sparse-matrix applications.

\section{Scientific Interests}
During my undergraduate studies I obtained a strong interest in the field of mathematics, logic and computer science. I am particularly interested in exploring applications of the above areas in the fields of biology, pharmacology and medicine. I am keen on continuing my academic coursework with a research oriented series of postgraduate studies.
 
\section{Notable Academic Projects}
\begin{longtable}{r|p{15cm}}
\textit{$9^{th}$ semester} & \large{Parallel Processing Systems}
\\& Exhaustive Speedup-scalability testing on LU Decomposition, focussing on two architecturally different implementations. Analysis aims to correlate the effects of algorithmic characteristics to parallelization prospects. Developed using the OpenMP API in a team of $4$ students.\\\multicolumn{2}{c}{}\\
\textit{$9^{th}$ semester} & \large{Software Engineering} 
\\& Design and implementation of three fully functional servers (Billing, Blocking, Forwarding), extending SIP on a given VOIP implementation, accompanied by Requirements Specification Documentation. Developed using Java, UML, MySQL.\\\multicolumn{2}{c}{}\\
\textit{$9^{th}$ semester} & \large{Design of Analog Electronic systems} 
\\& Analysis, design and implementation of a $10$W audio amplifier obligatory for academic course. Analysis,design and implementation of a class A-B audio amplifier and a class-D general purpose amplifier, regarding personal research interests, implemented with courses' mate.\\\multicolumn{2}{c}{}\\
\textit{$7^{th}$ semester} & \large{Database Systems} 
\\*& Design and implementation of a database management system simulating a complete realistic airport, following the MVC pattern. Developed using MySQL, PHP, HTML and Javascript in a team of $2$ students.\\\multicolumn{2}{c}{}
\end{longtable}

\section{Notable Academic Coursework}

\begin{longtable}[l]{r|p{15cm}}
\textit{$9^{th}$ semester} & \large{Neural Networks and Intelligent Systems}\\
& The course is focused on Neural network models and architectures, convergence and stability, error-correction (multilayer perceptrons and backpropagation algorithm) and competitive learning algorithms. Associative, recurrent and radial basis function networks. Computational intelligence and intelligent system applications.\\\multicolumn{2}{c}{}\\
\textit{$9^{th}$ semester} & \large{Advanced Topics in Database Systems}
\\& The main topics covered in this course include: Concurrency Control and Recovery in relational database management systems. Distributed databases. Object-oriented database systems. Temporal database systems. Spatial database systems. Data warehouses. Data Mining.\\\multicolumn{2}{c}{}\\
\textit{$8^{th}$ semester} & \large{Human-Machine Interaction}
\\& The course focuses on methodologies for user-centered systems' design, development and evaluation, covering issues like conceptual models for interaction, technologies, methods and tools for human-machine interaction design, etc. The main topics are human-computer interaction, interaction design methodologies, rich interaction, ubiquitous computing, and augmented realities.\\\multicolumn{2}{c}{}\\
\textit{$8^{th}$ semester} & \large{Advanced Topics in Computer Architecture}
\\& The course is focused on modern CPU organization: control unit and datapath, pipelined architectures, memory hierarchy organization, multi-stage pipelines with variable latencies, branch prediction, Very large Instruction Word (VLIW) architectures, Instruction Level Parallelism (ILP), superscalar pipelines, out-of-order execution. Examples of modern processors, hyperthreading (HT), Simultaneous Multithreading (SMT) and Multicore chips.\\\multicolumn{2}{c}{}\\
\textit{$7^{th}$ semester} & \large{Image and Video Analysis and Technology}
\\& The course is focused on processing, analysis, management, search and retrieval of digital images and digital video. Emphasis on image and video characteristics, 2-D sampling and transforms, quantization, coding, transmission of still and moving images, enhancement, (non-linear) filtering and image analysis, feature extraction, classification, summarization.\\\multicolumn{2}{c}{}\\
\end{longtable}

\section{Technical Skills}
\begin{tabular}{r|p{11cm}}
\textit{Programming Languages} & C/C++, Python, ML, Matlab, Prolog, SQL, Java, Haskell, HTML/CSS
\\&Assembly Languages: Intel $8085$/$8086$, AVR ASM\\\multicolumn{2}{c}{}\\
\textit{Operating Systems} & Linux/UNIX (Arch-primary OS, Manjaro, Debian, Ubuntu), Windows, Mac OS X\\\multicolumn{2}{c}{}\\
\textit{Tools} & Eclipse, Vim, Git, Pacman, Awesome(WM), \LaTeX\
\\&Parallel Programing: OpenMP, MPI
\\&Python Tools: Numpy, Matplotlib\\\multicolumn{2}{c}{}
\end{tabular}

\section{Achievements}
\begin{tabular}{r|p{11cm}}
\textit{June $2011$} & Highest Diploma Certification in Classical Guitar, Honors\\\multicolumn{2}{c}{}\\
\textit{May $2008$} & $1^{st}$ Place, National Greco-Roman Wrestling Student Championship\\\multicolumn{2}{c}{}
\end{tabular}


\section{Extracurricular Activities}
\begin{description}
\item[Cine-noisi] \hfill \\
Active member of the group of young people \emph{"Cine-noisi"}, who was granted by the \emph{Youth In Action EU program}, action $1.2$ Youth Initiatives, for the creation of the short movie \emph{"KANENAS"} (Translation: Nobody). The whole project had a social impact, referring to the loneliness of the elderly, through a learning process of the young people who implemented the project and who developed their personal and social competencies.
\item[Traditional Dances] \hfill 
\begin{itemize}
\item Current teacher at Traditional Dances Department, of National Technical University of Athens.
\item Active participation to the group "Chorovates".
\item Diploma with Honors, for graduation of the historical Pontian group "Argonaytoi-komnhnoi".
\end{itemize}
\item[Chess] \hfill \\
$1^{st}$ Place Diploma, Inter-Secondary School Championship.
\item[Athletic Activity] \hfill \\
8 Greco-Roman Wrestling Titles referring to Local Championships or Qualification Rounds towards National Championship.
\item[Remembering Dances] \hfill \\
Actively participated in the Exchange \emph{"Recordando Danzas"} as a part of the \emph{Youth in Action} program of the \emph{European Union} ($7$-$15$ June $2009$, Spain).
\item[First Poetry Prize-Premier Prix de Poesie] \hfill \\
$1^{st}$ Prize for the participation, with a team of classmates from secondary school, to the Greek poetry competition "Francophonie $2007$".
\end{description}

\section{Languages}
\begin{tabular}{r|p{11cm}}	
\textsc{Greek} & Mother tongue\\\multicolumn{2}{c}{}\\
\textsc{English} & Advanced Certificate-Michigan($2006$)
\end{tabular}

\end{document}
