%%%%%%%%%%%%%%%%%%%%%%%%%%%%%%%%%%%%%%%%%%%%%%%%%%%%%%%%%%%%%%%%%%%%%%%%
%%%%%%%%%%%%%%%%%%%%%% Simple LaTeX CV Template %%%%%%%%%%%%%%%%%%%%%%%%
%%%%%%%%%%%%%%%%%%%%%%%%%%%%%%%%%%%%%%%%%%%%%%%%%%%%%%%%%%%%%%%%%%%%%%%%

%%%%%%%%%%%%%%%%%%%%%%%%%%%%%%%%%%%%%%%%%%%%%%%%%%%%%%%%%%%%%%%%%%%%%%%%
%% NOTE: If you find that it says                                     %%
%%                                                                    %%
%%                           1 of ??                                  %%
%%                                                                    %%
%% at the bottom of your first page, this means that the AUX file     %%
%% was not available when you ran LaTeX on this source. Simply RERUN  %%
%% LaTeX to get the ``??'' replaced with the number of the last page  %%
%% of the document. The AUX file will be generated on the first run   %%
%% of LaTeX and used on the second run to fill in all of the          %%
%% references.                                                        %%
%%%%%%%%%%%%%%%%%%%%%%%%%%%%%%%%%%%%%%%%%%%%%%%%%%%%%%%%%%%%%%%%%%%%%%%%

%%%%%%%%%%%%%%%%%%%%%%%%%%%% Document Setup %%%%%%%%%%%%%%%%%%%%%%%%%%%%

\documentclass[10pt]{article}
\usepackage{calc}
\usepackage{comment}
\usepackage[utf8]{inputenc}


\makeatletter
\newlength{\bibhang}
\setlength{\bibhang}{1em} %1em}
\newlength{\bibsep}
 {\@listi \global\bibsep\itemsep \global\advance\bibsep by\parsep}
\newenvironment{bibsection}%
        {\begin{enumerate}{}{%
%        {\begin{list}{}{%
       \setlength{\leftmargin}{\bibhang}%
       \setlength{\itemindent}{-\leftmargin}%
       \setlength{\itemsep}{\bibsep}%
       \setlength{\parsep}{\z@}%
        \setlength{\partopsep}{0pt}%
        \setlength{\topsep}{0pt}}}
        {\end{enumerate}\vspace{-.6\baselineskip}}
%        {\end{list}\vspace{-.6\baselineskip}}
\makeatother

% Layout: Puts the section titles on left side of page
\reversemarginpar

%
%         PAPER SIZE, PAGE NUMBER, AND DOCUMENT LAYOUT NOTES:
%
% The next \usepackage line changes the layout for CV style section
% headings as marginal notes. It also sets up the paper size as either
% letter or A4. By default, letter was used. If A4 paper is desired,
% comment out the letterpaper lines and uncomment the a4paper lines.
%
% As you can see, the margin widths and section title widths can be
% easily adjusted.
%
% ALSO: Notice that the includefoot option can be commented OUT in order
% to put the PAGE NUMBER *IN* the bottom margin. This will make the
% effective text area larger.
%
% IF YOU WISH TO REMOVE THE ``of LASTPAGE'' next to each page number,
% see the note about the +LP and -LP lines below. Comment out the +LP
% and uncomment the -LP.
%
% IF YOU WISH TO REMOVE PAGE NUMBERS, be sure that the includefoot line
% is uncommented and ALSO uncomment the \pagestyle{empty} a few lines
% below.
%

%% Use these lines for letter-sized paper
\usepackage[paper=letterpaper,
            %includefoot, % Uncomment to put page number above margin
            marginparwidth=1.2in,     % Length of section titles
            marginparsep=.05in,       % Space between titles and text
            margin=0.8in,               % 1 inch margins
            includemp]{geometry}

%% Use these lines for A4-sized paper
%\usepackage[paper=a4paper,
%            %includefoot, % Uncomment to put page number above margin
%            marginparwidth=30.5mm,    % Length of section titles
%            marginparsep=1.5mm,       % Space between titles and text
%            margin=25mm,              % 25mm margins
%            includemp]{geometry}

%% More layout: Get rid of indenting throughout entire document
\setlength{\parindent}{0in}

\usepackage[shortlabels]{enumitem}

%% Reference the last page in the page number
%
% NOTE: comment the +LP line and uncomment the -LP line to have page
%       numbers without the ``of ##'' last page reference)
%
% NOTE: uncomment the \pagestyle{empty} line to get rid of all page
%       numbers (make sure includefoot is commented out above)
%
\usepackage{fancyhdr,lastpage}
\pagestyle{fancy}
%\pagestyle{empty}      % Uncomment this to get rid of page numbers
\fancyhf{}\renewcommand{\headrulewidth}{0pt}
\fancyfootoffset{\marginparsep+\marginparwidth}
\newlength{\footpageshift}
\setlength{\footpageshift}
          {0.5\textwidth+0.5\marginparsep+0.5\marginparwidth-2in}
\lfoot{\hspace{\footpageshift}%
       \parbox{4in}{\, \hfill %
                    \arabic{page} of \protect\pageref*{LastPage} % +LP
%                    \arabic{page}                               % -LP
                    \hfill \,}}

% Finally, give us PDF bookmarks
\usepackage{color,hyperref}
\definecolor{darkblue}{rgb}{0.08,0.02,0.4}
\hypersetup{colorlinks,breaklinks,
            linkcolor=darkblue,urlcolor=darkblue,
            anchorcolor=darkblue,citecolor=darkblue}

%%%%%%%%%%%%%%%%%%%%%%%% End Document Setup %%%%%%%%%%%%%%%%%%%%%%%%%%%%


%%%%%%%%%%%%%%%%%%%%%%%%%%% Helper Commands %%%%%%%%%%%%%%%%%%%%%%%%%%%%

% The title (name) with a horizontal rule under it
% (optional argument typesets an object right-justified across from name
%  as well)
%
% Usage: \makeheading{name}
%        OR
%        \makeheading[right_object]{name}
%
% Place at top of document. It should be the first thing.
% If ``right_object'' is provided in the square-braced optional
% argument, it will be right justified on the same line as ``name'' at
% the top of the CV. For example:
%
%       \makeheading[\emph{Curriculum vitae}]{Your Name}
%
% will put an emphasized ``Curriculum vitae'' at the top of the document
% as a title. Likewise, a picture could be included:
%
%   \makeheading[\includegraphics[height=1.5in]{my_picutre}]{Your Name}
%
% the picture will be flush right across from the name.
\newcommand{\makeheading}[2][]%
        {\hspace*{-\marginparsep minus \marginparwidth}%
         \begin{minipage}[t]{\textwidth+\marginparwidth+\marginparsep}%
             {\large \bfseries #2 \hfill #1}\\[-0.15\baselineskip]%
                 \rule{\columnwidth}{1pt}%
         \end{minipage}}

% The section headings
%
% Usage: \section{section name}
\renewcommand{\section}[1]{\pagebreak[3]%
    \hyphenpenalty=10000%
    \vspace{1.3\baselineskip}%
    \phantomsection\addcontentsline{toc}{section}{#1}%
    \noindent\llap{\scshape\smash{\parbox[t]{\marginparwidth}{\raggedright #1}}}%
    \vspace{-\baselineskip}\par}

% An itemize-style list with lots of space between items
\newenvironment{outerlist}[1][\enskip\textbullet]%
        {\begin{itemize}[#1,leftmargin=*]}{\end{itemize}%
         \vspace{-.2\baselineskip}}

% An environment IDENTICAL to outerlist that has better pre-list spacing
% when used as the first thing in a \section
\newenvironment{lonelist}[1][\enskip\textbullet]%
        {\begin{list}{#1}{%
        \setlength{\partopsep}{0pt}%
        \setlength{\topsep}{0pt}}}
        {\end{list}\vspace{-.4\baselineskip}}

% An itemize-style list with little space between items
\newenvironment{innerlist}[1][\enskip\textbullet]%
        {\begin{itemize}[#1,leftmargin=*,parsep=0pt,itemsep=3pt,topsep=0pt,partopsep=0pt]}
        {\end{itemize}}

\newenvironment{flist}[1][\enskip\textbullet]%
        {\begin{itemize}[#1,leftmargin=*,parsep=0pt,itemsep=3pt,topsep=3pt,partopsep=0pt]}
        {\end{itemize}}

% An environment IDENTICAL to innerlist that has better pre-list spacing
% when used as the first thing in a \section
\newenvironment{loneinnerlist}[1][\enskip\textbullet]%
        {\begin{itemize}[#1,leftmargin=*,parsep=0pt,itemsep=0pt,topsep=0pt,partopsep=0pt]}
        {\end{itemize}\vspace{-.6\baselineskip}}

% To add some paragraph space between lines.
% This also tells LaTeX to preferably break a page on one of these gaps
% if there is a needed pagebreak nearby.
\newcommand{\blankline}{\quad\pagebreak[3]}
\newcommand{\halfblankline}{\quad\vspace{-0.5\baselineskip}\pagebreak[3]}

\newcommand{\dates}[1]{\hfill {\small{\textsc{#1}}}}
% Uses hyperref to link DOI
\newcommand\doilink[1]{\href{http://dx.doi.org/#1}{#1}}
\newcommand\doi[1]{doi:\doilink{#1}}

% For \url{SOME_URL}, links SOME_URL to the url SOME_URL
\providecommand*\url[1]{\href{#1}{#1}}
% Same as above, but pretty-prints SOME_URL in teletype fixed-width font
\renewcommand*\url[1]{\href{#1}{\texttt{#1}}}

% For \email{ADDRESS}, links ADDRESS to the url mailto:ADDRESS
\providecommand*\email[1]{\href{mailto:#1}{#1}}
% Same as above, but pretty-prints ADDRESS in teletype fixed-width font
%\renewcommand*\email[1]{\href{mailto:#1}{\texttt{#1}}}

%\providecommand\BibTeX{{\rm B\kern-.05em{\sc i\kern-.025em b}\kern-.08em
%    T\kern-.1667em\lower.7ex\hbox{E}\kern-.125emX}}
%\providecommand\BibTeX{{\rm B\kern-.05em{\sc i\kern-.025em b}\kern-.08em
%    \TeX}}
\providecommand\BibTeX{{B\kern-.05em{\sc i\kern-.025em b}\kern-.08em
    \TeX}}
\providecommand\Matlab{\textsc{Matlab}}

%%%%%%%%%%%%%%%%%%%%%%%% End Helper Commands %%%%%%%%%%%%%%%%%%%%%%%%%%%

%%%%%%%%%%%%%%%%%%%%%%%%% Begin CV Document %%%%%%%%%%%%%%%%%%%%%%%%%%%%
\begin{document}

\makeheading{Zoi Paraskevopoulou}

\section{Personal Information}

% NOTE: Mind where the & separators and \\ breaks are in the following
%       table.
%
% ALSO: \rcollength is the width of the right column of the table
%       (adjust it to your liking; default is 1.85in).
%
\newlength{\rcollength}\setlength{\rcollength}{1in}%
%
\begin{tabular}[t]{@{}p{0.95in}p{1.3in}p{0.9in}p{2.5in}}
%\href{http://www.cse.osu.edu/}%
%     {Department of Computer Science and Engineering} & \\
%\href{http://www.osu.edu/}{The Ohio State University}
\textbf{Date of birth}: & 31 July 1990	  &  \textbf{Webpage}: & \href{http://zoep.github.io}{zoep.github.io} \\
\textbf{Gender}: & Female         			& \textbf{Email(s)}: & \email{zoe.paraskevopoulou@gmail.com}	\\
 \textbf{Citizenship}: & Greek          	    & & 	\email{zoi.paraskevopoulou@ens-cachan.fr }			
\end{tabular}

%\section{Objective}

%Insert text here if you want to
%\begin{innerlist}
%\item More information and auxiliary documents can be found at\\\url{http://www.tedpavlic.com/facjobsearch/}
%\end{innerlist}
\section{Education}

\textbf{Master's Degree}	 \dates{September 2014 to September 2015 (expected)} \medskip \\ 
\href{https://wikimpri.dptinfo.ens-cachan.fr/doku.php?id=start}{Master Parisien de recherche en Informatique}, École Normale Supérieure de Cachan, France 
\begin{flist}
\item[] Level: M2
\item[] Specialization: Logics and Semantics of Programs
\item[] Courses: \medskip \\ 
\newlength{\nlength}\setlength{\nlength}{2.2in}%
\begin{small}
\begin{tabular}[t]{@{}p{\nlength}p{\nlength}}
\begin{minipage}{3in}
\begin{innerlist}
    \item Foundations of proof systems		
	\item Linear logic and logical \\ paradigms of computation
	\item Automated deduction
	\item Abstract interpretation
\end{innerlist}
\end{minipage}	
&
\begin{minipage}{3in}
\begin{innerlist}
		
	\item Proof assistants
	\item Functional programming and  type systems
	\item Proofs of programs
	\item Semantics, languages and \\ algorithms for multicore programming

\end{innerlist}
\end{minipage}
\end{tabular}
\end{small}
\end{flist}
\bigskip
\textbf{Diploma} (5-year degree)	 \dates{September 2008 to September 2014} \medskip \\ 
\href{http://www.ece.ntua.gr}{School of Electrical and Computer Engineering}, National Technical University of Athens, Greece 
\begin{flist}
\item[] Majors: Computer Software, Computer Systems
\item[] Minors: Mathematics, Computer Networks
\item[] Thesis: \textit{A Coq Framework For Verified Property Based Testing}, Grade: 10/10
\item[] Thesis Committee: Nikolaos Papaspurou, Kostis Sagonas, Yannis Smaragdakis 
\item[] GPA: 8.4/10 
%\item[] Notable Courses: \medskip \\ 
%\newlength{\mylength}\setlength{\mylength}{2.2in}%
%\begin{small}
%\begin{tabular}[t]{@{}p{\mylength}p{\mylength}}
%\begin{minipage}{3in}
%\begin{innerlist}
%	\item Compilers
%	\item Programming Languages II
%	\item Cryptography
%	\item Parallel Processing Systems
%\end{innerlist}
%\end{minipage}	
%&
%\begin{minipage}{3in}
%\begin{innerlist}
%	\item Advanced Topics in Database Systems
%	\item Mathematical Logic for Computer Science
%	\item Applications of Logic in Computer Science
%	\item Operating Systems Laboratory
%\end{innerlist}
%\end{minipage}
%\end{tabular}
%\end{small}
\end{flist}
\bigskip
\textbf{General Lyceum} (Upper Secondary School) \dates{September 2005 to June 2008}\medskip \\ 
Geitonas School, Athens, Greece
\begin{flist}
\item[] GPA: 19.5/20
\end{flist}

\section{Research Experience}
\textbf{Research Internship} at INRIA Paris-Rocquencourt \dates{April 2014 to September 2014} 
 \begin{flist}
 	\item Topic: \textit{QuickChick:  A Coq Framework For Verified Property Based Testing }
	\item Advisor: Cătălin Hriţcu
	\item Team: \textsc{Prosecco}
 \end{flist}

\section{Workshop \\ Talks}
\textit{A Coq Framework For Verified Property-Based Testing (Extended Abstract)}. \\
 Zoe Paraskevopoulou, Catalin Hritcu, Maxime Dénès, Leonidas Lampropoulos and Benjamin C. Pierce. CoqPL 2015.
\medskip \\
\textit{QuickChick: Property-Based Testing for Coq}. \\
Maxime Dénès, Catalin Hritcu, Leonidas Lampropoulos, Zoe Paraskevopoulou and Benjamin C. Pierce. The 6th Coq Workshop. July 2014.

\section{Scholarships and Awards}

Selected for \textbf{scholarship} for attending PLMW at POPL 2015. \dates{2014}\medskip \\ 
\textbf{INRIA-MPRI Scholarship} \dates{2014}\\
1 year scholarship for attending the MPRI program. \medskip \\
\textbf{Scholarship} for attending Applied Functional Programming in Haskell  \dates{2013}\\ Summer School, Utrecht University,  Netherlands. 

\medskip
\section{Other Courses}
\textbf{Summer School} on \href{http://www.cs.uu.nl/wiki/USCS}{Applied Functional Programming in Haskell}  \dates{August 2013} \\
Utrecht University, Netherlands. \medskip  \\
Certificates of accomplishment  for the following \textbf{Online Courses}:
\begin{flist}
\item Cryptography I	\dates{March 2013} \\
provided by Standford University through Coursera Inc.
\item Software as a Service \dates{November 2012} \\
provided by BerkeleyX through edX	
\end{flist}

\section{Notable Student Projects}
\begin{innerlist}
\item[]	\href{https://github.com/zoep/jebus}{\textbf{Lambda Calculus Interpreter}}	\dates{November 2013}			\\
An interpreter for a typed lambda calculus variant featuring let and let-rec definitions, if-then-else construct, pairs, various arithmetic, boolean and relative operators, type inference and let-polymorphism. Implemented in Haskell in a team of 2 students.\\[-0.1cm]
\item[]	\href{https://github.com/zoep/alpaca}{\textbf{Llama Compiler}}	\dates{October 2013}			\\
A compiler for an OCaml-like language with pattern matching, type inference, higher-order functions and user defined data types. The compiler performs control flow graph, peephole and tail call optimizations. Developed in OCaml in a team of 3 students.\\[-0.1cm]
\item[] \textbf{Advanced Topics in Database Systems Project} \dates{March 2013} \\
A bibliographic report about security and cryptography in database systems, written in a team of 2 students.\\[-0.1cm]
\item[] \href{https://github.com/zoep/ECC-OCaml}{\textbf{Cryptography Project}}	\dates{January 2013} \\
A library implementing basic operations on elliptic curves over prime fields, Elliptic Curve digital signature and Diffie-Hellman key exchange algorithms. Developed in Ocaml in a team of 2 students.\\[-0.1cm]
\item[] \href{http://zoep.github.io/db-sec.pdf}{\textbf{Database Systems Project}}		\dates{February 2012} \\
Design and implementation of a database management system for a fictional airport, following the MVC pattern. Developed using MySQL, PHP, HTML and Javascript in a team of 2 students.
\end{innerlist}

\section{Computer Skills}
\begin{innerlist}
\item[] \textbf{Proof Assistants} \\ Coq, \textsc{SSReflect} extension
\item[] \textbf{Programming Languages} \\ OCaml, Haksell, Prolog, C, Erlang, Unix Shell Scripting, Assembly (x86, AVR) 
\item[] \textbf{Other Tools and Frameworks} \\ Git, \LaTeX, Gnuplot, VIM, Emacs, Frama-C
\end{innerlist}

\section{Interests}
Programming languages theory and implementation, computer security, static analysis, software testing and verification, formal methods, logic, cryptography 

\section{Languages}
\begin{innerlist}
	\item[] \textbf{Greek} Mother Tongue
	\item[] \textbf{English} Proficient speaking and writing skills 
	\item[] \textbf{French} Elementary speaking and writing skills 
\end{innerlist}
\section{Other \\ Activities}

Music studies at the National Conservatory of Athens.
\begin{flist}
\item[] \textbf{Piano}	\dates{September 2008 to present} 
\item[] \textbf{Choral Conducting} \dates{September 2012 to June 2014} 
\item[] \textbf{Theory of Harmonization} \dates{September 2011 to June 2014} 
\item[] \textbf{Music Theory} 	\dates{September 2010 to June 2011} \\
\end{flist}
\end{document}